We present our work towards building a multi-document summarizer, focusing on the guided summarization problem.  By making use of information common to document sets belonging to a same category, we improve the quality of the content extracted in the final summary. This simple property is widely applicable in multi-document summarization tasks, and can be encapsulated by the concept of \emph{category-specific importance (CSI)}.  Our experiments show that CSI is a valuable metric to aid sentence selection in extractive summarization tasks. We operationalize the computation \emph{CSI} of sentences through the introduction of two new features that can be computed without needing any external knowledge.  We have incorporated these features into a simple, freely available, open-source extractive summarization system, called SWING.  In the recent TAC 2011 guided summarization task, SWING outperformed all other participant summarization systems as measured by automated ROUGE measures. 
% Min: I don't know what this sentence is supposed to add, but left it in anyways.
Finally, we generalize our approach, showing that performing automatic categorization of document sets, potentially improves summarization performance. 
%We manually examined the summaries as part of error analysis, and identified the need to normalize numerical information in summaries.
