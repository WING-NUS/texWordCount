To gain deeper insight on category specific information, we try to examine the improvements TopicSumm+CSI has over TopicSumm. We manually studied the differences in the summaries generated by the two different systems.
In the test topics, we found that the CSI version selected alternative sentences in 14 out of the 44 topics. The categories \emph{Accidents}, \emph{Attacks}, and \emph{Investigations} have 3 replacements each while \emph{Health} and \emph{Endangered Resources} have 1 and 4 replacements, respectively.  Less important a phenomenon is that the summary sentences were re-ordered in 10 instances, resulting in minor changes in ROUGE scores, as the last sentence is  trimmed to keep the summary length to 100 words.  The changes made by CSI in the selection are thus frequent, altering some summaries in a substantial way, made evident by the change in ROUGE score. 

To illustrate the utility of leveraging on category specific information, differences between both the configurations for a topic in \emph{Accidents} category are provided in Figure~\ref{figure:csi_summaries}. 
%Summary sentences are not coherent since they are extracted from various parts of documents from multiple sources. This effects the readability of the summaries, as our methods are only content specific and doesn't use any language processing techniques like discourse analysis and anaphora resolution. The changes caused by CSI are \textit{italicized} in the summaries. Two sentences are replaced in the CSI configuration resulting in a more informative summary. 
The `$-$' sign represents that the sentence is excluded and `$+$' sign shows that the sentence is included in TopicSumm+CSI configuration. The first replaced sentence has more category specific words like ``warning'', ``earthquake'', ``killed'', ``people'' compared to the original sentence ``death'', ``buried'', and offers more information.
% ZH3: why {``warning'', ``earthquake'', ``killed'', ``people''} are category-specific and {``death'', ``buried''} are not? The first four are not in Table 4 so they are not very convincing...

\begin{figure}[h]
    \centering
\fbox{
\begin{minipage}{.8\linewidth}
\textbf{TopicSumm:}\\
$-$ \textit{The death toll could rise as thousands are still buried in debris and many are reported missing}.\\
$-$ \textit{Therefore, the relevant sectors and personnel should pay attention to disaster prevention.}\\ 
\textbf{TopicSumm+CSI:}\\
$+$ \textit{Chinese authorities did not detect any warning signs ahead of Monday's earthquake that killed more than \textbf{8,600 people}.}\\
$+$ \textit{Xinhua said \textbf{8,533 people} had died in Sichuan alone, citing the local government.}
\end{minipage}
}
\caption{Difference in summaries for the topic ``Earthquake in Sichuan'', from \emph{Accidents} category.}
\label{figure:csi_summaries}
\end{figure}

%Also, it is observed that CSI features may sometimes negatively effect the summaries too. 
When we compute CSI scores for sentences, we do not look at whether sentences have redundant category-specific information and whether all aspects of the category are covered by the selected sentences.
For example, we observed that the second replaced sentence repeats the information in previous sentences of the summary but still gets selected into the summary due to the presence of more category specific words. In the future, we plan to use category specific statistics in a more organized way to remove category-specific redundancy (akin to MMR) and to include all aspects of information in the summary. 
%category aspect words
%disperancies in numerical info

Numerical information in a topic, such as casualties, temporal markers, monetary damages can also conflict within documents in a set on a topic, as they are compiled by different sources and at different points of time. For example, the number of casualties (\textbf{bolded} in summaries) is specified as 8,533 and more than 8,600 in different sentences. While any of these sentences could be selected into a summary due to similar content words, the corresponding model summary has only the most updated information (12,000 people). As a result the evaluation scores are dropped although the summarizer picks an informative sentence. This highlights the need to normalize such numerical information in the summaries which are important in categories like accidents and attacks where quantitative information is key.

\begin{figure}[h]
\centering		
\includegraphics[width=0.8\textwidth]{csi_category_r2.png}
\caption{ROUGE-2 scores at category level for TopicSumm and TopicSumm+CSI configurations.}
\label{fig:csi-r2}
\end{figure}
 
%Out study also shows that certain types of numerical information are very important for certain categories.
%Carry out numerical information for csi if time permits
We further observed that the difficulty to summarize a topic may vary upon its category.  ROUGE-2 performance broken done by categories is shown in Figure~\ref{fig:csi-r2}, revealing that the topics in \emph{Endangered Resources} and \emph{Health} are the most difficult to summarize. 
%Similar observation is made by the organizers at TAC~\cite{tac:overview}. The average ROUGE score of all the participant systems in TAC 2011 for category `Endangered Resources' , `Health' is significantly less than other categories. 
We believe that the larger presence of subjective aspects (\textit{How, Why, Threats}) in both categories increases the difficulty for automatic summarizers to recognize relevant information.
The topics in other three categories are easier to summarize: we note that the improvement on \emph{Attacks} and \emph{Accidents} with the CSI features are more pronounced than in the remaining categories.  When we look at the aspects defined by TAC for both \emph{Attacks} and \emph{Accidents}, we notice that seven of their aspects overlap, as shown in Table~\ref{table:2_aspects}.  This suggests that the more general aspects a category has, the easier it is to compute its category-specific information.
In our future work, we plan to look at how we can utilize general versus specific aspects to improve our model of CSI.


\begin{table}[h]
\centering
\begin{center}
\begin{tabular}{l||l}
\textbf{Category} & \textbf{Aspects}  \\ \hline
        Accidents & WHAT, WHEN, WHERE, WHY, WHO\_AFFECTED, \\
                  & DAMAGES, COUNTERMEASURES \\
          Attacks & WHAT, WHEN, WHERE, PERPETRATORS, WHY, \\
                  & WHO\_AFFECTED, DAMAGES, COUNTERMEASURES \\
\end{tabular}
\end{center}
\caption{Aspects for categories \emph{Accidents} and \emph{Attacks} defined in TAC. Seven aspects overlap in these two categories.}
\label{table:2_aspects}
\end{table}

%ROUGE-2 scores are increased by 3.4\%, 6.3\%, 1.9\% while ROUGE-SU4 scores are improved by 1.6\%, 5.2\%,1.7\% with the addition of CSI features in `Attacks',`Accidents' and `Investigations' respectively. However the changes are not convincing in both `Health' and `Endangered Resources' categories.
%\begin{figure}[h]
%\centering		
%\includegraphics[width=260px,height=190px]{csi_category_rsu4.png}
%\caption{ROUGE-SU4 scores at category level for TopicSumm and TopicSumm+CSI configurations.}
%\label{fig:csi-rsu4}
%\end{figure}
%include graphs instead of tables
%\begin{table*}[h]
%\centering
%\begin{center}
%\begin{tabular}{l||l|l||l|l|}
%& \multicolumn{2}{|c|}{ROUGE2}& \multicolumn{2}{|c|}{ROUGE-SU4}\\ \hline
%Category & SWING & SWING+CSI & SWING & SWING+CSI \\ \hline
%Accidents             & 0.1489   & 0.15896   & 0.173  & 0.1825\\
%Attacks      		 & 0.1475 & 0.1520   & 0.1867 & 0.1892 \\ 
%Health              & 0.1343   & 0.1340    & 0.1627 & 0.1639 \\
%Resources 			&  0.0838  &  0.0848   & 0.1217 & 0.1228\\
%Investigations 		& 	0.15125& 0.1548    & 0.1775 & 0.1779\\
%\end{tabular}
%\end{center}
%\caption{Category level distribution of ROUGE Scores}
%\label{table:csi_phrases}
%\end{table*}
%factual information
